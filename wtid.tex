\documentclass[12pt]{scrartcl}
\usepackage{amsmath}
\usepackage{mathrsfs}
\usepackage{amsthm}
\usepackage{xspace}
\usepackage[libertine]{newtxmath}
%\usepackage{newtxtext}
%\usepackage{physics}
\usepackage{tgtermes,tgheros,tgcursor}
\usepackage{type1cm}
\usepackage{physics}

\newcommand{\del}{\partial}
\newcommand{\Ncal}{\mathcal{N}}
\newcommand{\Zb}{\mathbb{Z}}
\newcommand{\Ocal}{\mathcal{O}}
\newcommand{\phih}{\hat{\phi}}
\newcommand{\Lcal}{\mathcal{L}}
\newcommand{\psib}{\bar{\psi}}

\newcommand{\strong}[1]{{\sffamily \bfseries #1}}

\begin{document}
\begin{center}
{\bfseries \sffamily \LARGE Ward-Takahashi Identities in the Euclidean Path-Integral Formalism}  \\

\vspace{0.5cm}
\noindent {\bfseries \sffamily \Large Satoshi Yamaguchi}  
\end{center}

\section{Introduction}
We would like to derive Ward-Takahashi(WT) identities within the Euclidean path-integral formalism. The purpose of this note is to fix the conventions including signs and $\pm i$.  This kind of derivation does not often appear in the standard textbooks of quantum field theory.  There are nice explanation in \cite{Polchinski:1998rq}. However, it focuses on two dimensions and the standard two-dimensional CFT convention is different from the standard convention in higher dimensions.

This note also includes the WT identities for discrete symmetries that do not appear standard textbooks.  Such WT identities appears in \cite{Gaiotto:2014kfa}.  However, it is treated as a kind of axioms.  In this note, we would like to ``derive'' the identities from the Euclidean path-integral formalism.

Let us prepare the notations used in this note.  We consider $d$-dimensional quantum field theories in the Euclidean path-integral formalism.
Let $\phi(x)$ the dynamical fields of the theory and $S[\phi]$ the Euclidean action. $\phi(x)$ may have multiple components.  The partition is expressed as
\begin{align}
    Z=\int D\phi \exp(-S[\phi]).
\end{align}
The expectation value of an operator $X$ is given by
\begin{align}
    \expval{X}:=\frac{1}{Z}\int D\phi X \exp(-S[\phi]).
\end{align}
The symmetry is the transformation $\phi(x)\to \phi'(x)$ which leaves action invariant:
\begin{align}
    S[\phi']=S[\phi].
\end{align}
In this note, we assume the integral measure $\int D\phi $ is always invariant, for simplicity.

\section{Continuous symmetry}
For a continuous symmetry, it is convenient to consider infinitesimal transformation:
\begin{align}
    \phi(x)\to \phi'(x)=\phi(x)+\epsilon \Delta \phi(x).
\end{align}
Here $\epsilon$ is the infinitesimal parameter of the transformation, and $\Delta \phi(x)$ is a finite local operator.
The invariance of the action implies
\begin{align}
    \int d^dx\Delta \phi(x)\fdv{S}{\phi(x)}=0.
\end{align}

The strategy to obtain the WT identity is to promote the infinitesimal constant parameter $\epsilon$ to an arbitrary infinitesimal function $\epsilon(x)$.  Then the following identity holds:
\begin{align}
    &\int D\phi \exp(-S[\phi])=
    \int D\phi' \exp(-S[\phi'])=
    \int D\phi \exp(-S[\phi]) (1-\delta S[\phi]),\\
    &\delta S[\phi]:=S[\phi']-S[\phi].
\end{align}
In the first step, we just change the letter.  In the second step, we use the invariance of the measure.
This identity is rewritten as
\begin{align}
    \expval{\delta S[\phi]}=0.
    \label{contWTid0}
\end{align}

Let us look closely at $\delta S$.  Since $\delta S=0$ when $\del_{\mu}\epsilon=0$,  $\delta S$ can be written as a linear functional of $\del_{\mu}\epsilon$:
\begin{align}
    \delta S=\int d^dx \fdv{S}{\phi(x)}\epsilon(x)\Delta\phi(x)
    =:-\int d^d x\del_{\mu}\epsilon(x) J^{\mu}(x)=
    \int d^dx \epsilon(x)\del_{\mu}J^{\mu}(x).\label{defcurrent}
\end{align}
Here we introduce the current $J^{\mu}(x)$.  We define this current such that the spacial components of $J^{\mu}$ are identical to the ones in the standard Lorentzian formalism.

The identity \eqref{contWTid0} implies
\begin{align}
    \expval{\del_{\mu}J^{\mu}}=0.
\end{align}
This is the quantum version of the conservation of the current and called Ward-Takahashi (WT) identity.

We can perform the same trick when other operator $X$ is inserted.  $X$ includes some local operators and (or) defects.  In this case, the story is the same if we consider $\epsilon(x)$ that vanishes at the place where $X$ is inserted, since $X'=X$ in this case.  We obtain a bit more general WT identity:
\begin{align}
    \expval{\del_{\mu}J^{\mu}(x) X}=0,\ (x \text{ does not coincide with the points where $X$ is inserted.})
\end{align}
In the following discussion, we always assume that $x$ does not coincide with the points where $X$ is inserted, without indication.

Next, let us consider a local operator $\Ocal(x)$.  Suppose $\Ocal$ transformed as
\begin{align}
    \Ocal(x)\to \Ocal'(x)=\Ocal(x)+\epsilon \Delta \Ocal(x),
\end{align}
by the symmetry transformation that we are considering.  When promoting $\epsilon$ to a function, you may worry that the transformation of $\Ocal$ may depend not only on $\epsilon(x)$ but also on the derivatives of $\epsilon(x)$.  To avoid this, we consider $\epsilon(x)$ that is a constant in a small open set including $x_1$.  Such $\epsilon(x)$ is arbitrary enough.  Then, the transformation of $\Ocal(x_1)$ is written as
\begin{align}
    \Ocal(x_1)\to \Ocal'(x_1)=\Ocal(x_1)+\epsilon(x_1) \Delta \Ocal(x_1).
\end{align}
By the same strategy as previous ones, we obtain the identity:
\begin{align}
    \int D\phi \Ocal(x_1)\exp(-S[\phi])X
    =\int D\phi' \Ocal'(x_1)\exp(-S[\phi'])X\nonumber\\
    =\int D\phi (\Ocal(x_1)+\epsilon(x_1)\Delta \Ocal(x_1))(1-\delta S)\exp(-S[\phi]).\nonumber\\
    \expval{\Ocal(x_1)\delta S\, X}
    =\expval{\epsilon(x_1)\Delta \Ocal(x_1)X} \label{tempWT0}
\end{align}


The left-hand side of this identity can be rewritten by using the definition of the current \eqref{defcurrent} as
\begin{align}
    \mathrm{l.h.s.}=\int d^dx \epsilon(x)\expval{\del_{\mu}J^{\mu}(x)\Ocal(x_1)X}.
\end{align}
On the other hand, the right-hand side is rewritten as
\begin{align}
    \mathrm{r.h.s.}=\int d^dx \epsilon(x)\delta^d(x-x_1)\expval{\Delta \Ocal(x_1)X}.
\end{align}
Thus, we obtain the WT identity:
\begin{align}
    \expval{\del_{\mu}J^{\mu}(x) \Ocal(x_1)X}=\delta^d(x-x_1)\expval{\Delta \Ocal(x_1)X}.
    \label{contWTid2}
\end{align}

Next, let us consider the charge.  Let us take a region $D$ which includes $x_1$ but does not include the points where $X$ is inserted. We integrate both sides of \eqref{contWTid2} and use Gauss's theorem:
\begin{align}
    \int_{\Sigma}dS_{\mu}\expval{J^{\mu}(x)\Ocal(x_1)X}=\expval{\Delta \Ocal(x_1)X},\label{tempWT1}
\end{align}
where $\Sigma$ is the boundary of $D$.  We introduce the charge on $\Sigma$:
\begin{align}
    Q(\Sigma):=-i\int_{\Sigma}dS_{\mu} J^{\mu}(x).\label{charge}
\end{align}
Then \eqref{tempWT1} is rewritten as
\begin{align}
    \expval{iQ(\Sigma)\Ocal(x_1)X}=
    \expval{\Delta \Ocal(x_1)X}.
    \label{tempWT2}
\end{align}

Finally, we introduce the symmetry defect.  The identity \eqref{tempWT2} reads
\begin{align}
    \expval{(1+i\epsilon Q(\Sigma))\Ocal(x_1)X}=\expval{\Ocal(x_1)X}.
\end{align}
So, it is convenient to define symmetry defect
\begin{align}
    U_{\epsilon}(\Sigma):=
    1+i\epsilon Q(\Sigma).
\end{align}
Then, the WT identity can be written as
\begin{align}
    \expval{U_{\epsilon}(\Sigma)\Ocal(x_1) X}=\expval{\Ocal'(x_1)X}.
\end{align}
This final form will be extended to the finite transformation, as well as the discrete symmetry.

\subsection{Relation to the gauge field}

The current is also expressed by the functional derivative of the action by the gauge field.  Here, we want to fix the convention.

Let us consider $U(1)$ global symmetry.   We introduce the background gauge field $A_{\mu}(x)$.  The action is denoted by $S[\phi,A]$.
The gauge transformation is given by
\begin{align}
    \phi'(x)=\phi(x)+\epsilon(x)\Delta \phi(x),\quad
    A'_{\mu}(x)=A_{\mu}(x)+\del_{\mu}\epsilon(x).
\end{align}
The action is invariant under this gauge transformation:
\begin{align}
    S[\phi',A']=S[\phi,A].
\end{align}
This relation reads
\begin{align}
    \int d^dx\left[
        \fdv{S}{\phi}\epsilon(x)\Delta \phi(x)
        +\fdv{S}{A_{\mu}(x)}\del_{\mu}\epsilon(x)
    \right]=0.
\end{align}
By comparing this equation with \eqref{defcurrent}, we obtain
\begin{align}
    J^{\mu}(x)=\fdv{S}{A_{\mu}(x)}.
\end{align}

\section{Discrete symmetry}
Let us pick up a group element $g$. Its action on the local operator $\Ocal(x)$ is denoted by
\begin{align}
    \Ocal(x)\to \Ocal^g(x).
\end{align}
Again, let us pick up a point $x_1$ that does not coincide with the points where $X$ is inserted.  We also choose a region $D$ including $x_1$.  The boundary of $D$ is denoted by $\Sigma$.  Let us consider the following change of variables in the path-integral:
\begin{align}
    \phih(x)=
    \begin{cases}
        \phi^{g^{-1}}(x),& (x\in D)\\
        \phi(x),& (x\notin D)
    \end{cases}.
\end{align}
We can derive the following identity:
\begin{align}
    &\expval{\Ocal^{g}(x_1) X}
    =\frac{1}{Z}\int D\phi \Ocal^{g}(x_1) X \exp(-S[\phi])
    =\frac{1}{Z}\int D\phih \widehat{\Ocal^{g}}(x_1) X \exp(-S[\phih])\nonumber\\    
    &\qquad =\frac{1}{Z}\int D\phi \Ocal(x_1) X \exp(-S[\phih]).
    \label{tempWT3}
    =\expval{U_{g}(\Sigma)\Ocal(x_1)X}
\end{align}
We define the symmetry defect $U_{g}(\Sigma)$ by
\begin{align}
    U_{g}(\Sigma):=\exp(-S[\phih]+S[\phi]).\label{symmetryoperator}
\end{align}
Then we can express the WT identity \eqref{tempWT3} as
\begin{align}
    \expval{\Ocal^{g}(x_1)X}
    =\expval{U_{g}(\Sigma)\Ocal(x_1)X}.
\end{align}

\section{Examples}
\subsection{$d$-dimensional Dirac fermion}
The action for the Dirac fermion is
\begin{align}
    S=\int d^d x \left(i \psib \gamma^{\mu}\del_{\mu}\psi+im \psib\psi\right).
\end{align}
This action has the vector $U(1)$ symmetry:
\begin{align}
    \delta \psi=i\epsilon \psi
    \delta \psib=-i\epsilon \psib
\end{align}
The infinitesimal transformation is
\begin{align}
    \delta\psi=i\epsilon\psi,\quad 
    \delta\psib=-i\epsilon\psib.
\end{align}

The current for this symmetry is obtained as
\begin{align}
    J^{\mu}=\psib \gamma^{\mu} \psi.
\end{align}
Therefore, the symmetry operator for an infinitesimal transformation is given by
\begin{align}
    U_{\epsilon}(\Sigma)=\exp\left(
        i\epsilon Q(\Sigma)
    \right)=\exp\left(
        \epsilon\int_{\Sigma}dS_{\mu}J^{\mu}
    \right),\quad
    Q(\Sigma)=-i\int_{\Sigma}dS_{\mu}J^{\mu}.
\end{align}
This expression is valid even when $\epsilon$ is replaced by a finite parameter $\alpha$:
\begin{align}
    U_{\alpha}(\Sigma)=\exp\left(
        \alpha\int_{\Sigma}dS_{\mu}J^{\mu}
    \right),
\end{align}
as one can check from the expression 
\eqref{symmetryoperator}.


\subsection{$d$-dimensional complex scalar field}


\section{2D CFT convention}


\appendix
\section{Currents in the Lorentzian signature}
Let us consider $d$-dimensional QFT with the fundamental field $\phi(x)$ and the Lorentzian action:
\begin{align}
    S_{L}=\int d^d x \Lcal(\phi,\del \phi).
\end{align}
Suppose this theory is invariant under the transformation:
\begin{align}
    \phi(x)\to \phi'(x)=\phi(x)+\epsilon \Delta \phi(x).
\end{align}
The action is invariant under this transformation.  Or equivalently, the variation of $\Lcal$ can be written with a vector field $K^{\mu}$ as
\begin{align}
    \delta \Lcal = \epsilon \del_{\mu}K^{\mu}(x).
\end{align}
Then, the Noether current is given by
\begin{align}
    J^{\mu}(x)=\pdv{\Lcal}{(\del_{\mu}\phi(x))} \Delta \phi(x)-K^{\mu}(x).
\end{align}
When promoting the $\epsilon$ to a function $\epsilon(x)$, the variation of the action can be written as
\begin{align}
    \delta S_{L}=\int d^d x \del_{\mu}\epsilon(x) J^{\mu}(x)=-\int d^d x \epsilon(x) \del_{\mu}J^{\mu}(x).
\end{align}
Actually
\begin{align}
    \delta S_{L}
    =&\int d^d x \left[
        \pdv{\Lcal}{\phi(x)}\epsilon(x)\Delta \phi(x)
        +\pdv{\Lcal}{(\del_{\mu}\phi(x))}\del_{\mu}(\epsilon(x)\Delta \phi(x))
    \right]\nonumber\\
    =&\int d^d x \left[\epsilon(x)\left(
        \pdv{\Lcal}{\phi(x)}\Delta \phi(x)
        +\pdv{\Lcal}{(\del_{\mu}\phi(x))}\epsilon(x)\del_{\mu}\Delta \phi(x)
        \right)
        +\del_{\mu}\epsilon(x)\pdv{\Lcal}{(\del_{\mu}\phi(x))}\Delta \phi(x)
    \right]\nonumber\\
    =&\int d^d x \left[\epsilon(x)\del_{\mu}K^{\mu}(x)
        +\del_{\mu}\epsilon(x)\pdv{\Lcal}{(\del_{\mu}\phi(x))}\Delta \phi(x)
    \right]\nonumber\\
    =&\int d^d x \del_{\mu}\epsilon(x) J^{\mu}(x).\nonumber\\
\end{align}

For a field configuration satisfying the equation of motion, $\delta S_{L}=0$ for an arbitrary variation $\delta \phi$.  Therefore, $\del_{\mu}J^{\mu}=0$ for such a configuration.

By a naive Wick rotation, $S_{L}$ is related to the Euclidean action $S_{E}$ by
\begin{align}
    iS_{L}=-S_{E}.
\end{align}
Taking the Euclidean time $x^d=ix^0$ into account, we get the relation
\begin{align}
    \delta S_{E}=-\int d^d x \del_{\mu}\epsilon(x) J^{\mu}(x)=\int d^d x \epsilon(x) \del_{\mu}J^{\mu}(x),
\end{align}
which coincide with \eqref{defcurrent}.  The ``time'' component of the current is related as
\begin{align}
    J^{d}=iJ^{0}.\label{timecomponent}
\end{align}
The charge $Q$ on a time slice is defined as
\begin{align}
    Q=\int_{\text{space}} d^{d-1}x J^{0}.
\end{align}
Taking \eqref{timecomponent} into account, one finds that the expression of charge \eqref{charge} is appropriate.

\begin{thebibliography}{9}
%\cite{Polchinski:1998rq}
\bibitem{Polchinski:1998rq}
J.~Polchinski,
``String theory. Vol. 1: An introduction to the bosonic string,''
doi:10.1017/CBO9780511816079
%515 citations counted in INSPIRE as of 05 May 2022
%\cite{Gaiotto:2014kfa}
\bibitem{Gaiotto:2014kfa}
D.~Gaiotto, A.~Kapustin, N.~Seiberg and B.~Willett,
%``Generalized Global Symmetries,''
JHEP \textbf{02} (2015), 172
doi:10.1007/JHEP02(2015)172
[arXiv:1412.5148 [hep-th]].
%616 citations counted in INSPIRE as of 05 May 2022
\end{thebibliography}

\end{document}